\documentclass[10pt]{article}
\usepackage[letterpaper,margin=1in]{geometry}
\usepackage{amsmath, amsthm, amssymb, amsthm}
\usepackage{mathtools}
\usepackage{csquotes}
\usepackage{hyperref}
\usepackage{comment}
\usepackage{fontawesome}
\usepackage[dvipsnames]{xcolor}
\usepackage[framemethod=tikz]{mdframed}

\mathtoolsset{showonlyrefs}

% Commands
\newcommand*{\indicator}[1]{\mathds{1}\left\{#1\right\}}
\newcommand*{\prob}[1]{\mathbb{P}\left[#1\right]}
\newcommand*{\expec}[1]{\mathbb{E}\left[#1\right]}
\newcommand*{\var}[1]{\mathrm{Var}\left[#1\right]}
\newcommand*{\conditioned}{\,\middle|\,}
\newcommand*{\eqcomment}[2][\qquad]{#1 &&\text{(#2)}}
\newcommand*{\bigslant}[2]{{\raisebox{.2em}{$#1$}\left/\raisebox{-.2em}{$#2$}\right.}}
\newcommand*{\inv}[1]{#1^{-1}}

% Shorthand
\newcommand*{\be}{\begin{equation}}
\newcommand*{\ee}{\end{equation}}

% Theorem
\newtheorem{theorem}{Theorem}
\newtheorem{lemma}{Lemma}
\newtheorem{corollary}{Corollary}
\newtheorem{proposition}{Proposition}
\newtheorem{conjecture}{Conjecture}
\newtheorem{hypothesis}{Hypothesis}
\newtheorem{question}{Question}
\newtheorem{observation}{Observation}
\newtheorem{example}{Example}
\newtheorem{claim}{Claim}
\newtheorem{definition}{Definition}
\newtheorem{remark}{Remark}

% Symbols
\newcommand*{\R}{\mathbb{R}}
\newcommand*{\RP}{\mathbb{R}P}
\newcommand*{\Z}{\mathbb{Z}}
\newcommand*{\C}{\mathbb{C}}
\newcommand*{\N}{\mathbb{N}}
\newcommand*{\Q}{\mathbb{Q}}
\newcommand*{\F}{\mathbb{F}}
\newcommand*{\eps}{\varepsilon}

% Miscellaneous
\newcommand*{\ignore}[1]{}
\newcommand*{\gaussian}[1]{\mathcal{N}\left(#1\right)}

\newmdtheoremenv{ideabox}{Idea}
\newcommand{\gb}{Gr\"{o}bner }
\newcommand*{\ideal}[1]{\langle#1\rangle}

\setlength{\parindent}{0pt}

\begin{document}

\begin{center}\sc\Large
\gb Bases
\end{center}

This is just meant to be a thought dump on \gb bases. My goal is for the following discussion to be such that it would lead one to the definition of a \gb basis. I'm not going to talk about monomial ordering. This discussion is for someone who has already done division in $k[x]$, and is familiar with the division algorithm in $k[x_1, \ldots, x_n]$. \\

Let's start with a simple example to show that there is a problem with the extended division algorithm. Let's try to see if $f = xy^2 - x$ is divisible by $(f_1 = xy - 1, f_2 = y^2 - 1)$\footnote{The $(\cdot, \cdot, \ldots)$ notation is meant to denote ordered tuples.} If we followed the order $f_1, f_2$, then we'd find that
\[
xy^2 - x = y(xy-1) + (-x + y),
\]
giving us a remainder of $y-x$. However, if we tried to divide by $(f_2, f_1)$, we get
\[
xy^2 - x = x(y^2-1) + 0,
\]
giving us a remainder of 0. Now, this non-uniqueness of remainder is clearly something we'd like to remedy. Trying to suggest that $(f_2, f_1)$ is somehow a better order than $(f_1, f_2)$ is a non-starter because we can easily construct an example where choosing $(f_2, f_1)$ would give us a non-zero remainder, while choosing $(f_1, f_2)$ would give us a remainder of 0. \\

\begin{ideabox}
Suppose we are trying to divide by $\{f_i\}$.\footnote{The $\{a_i\}$ notation just means that we have a finite set such as $\{a_1, \ldots, a_r\}$, and we are not referencing $r$ because it is just not important for our discussion.} If we find $\{g_i\}$  such that, for each $g_i$,
\[
g_i = \sum h^{(i)}_jf_j,
\]
and it turns out that dividing by $g_i$ is easier, then we are in better shape. This is because if we use $\{g_i\}$ to divide, it is as good as dividing by $\{f_i\}$. \textbf{This immediately suggests that we employ the language of ideals.}
\end{ideabox}

In the language ideals, checking if $\{f_i\} | f$ is equivalent to checking if $f \in \ideal{\{f_i\}}$. However, even with the new language, we haven't made any progress. $\{f_i\}$ is the generating set of the new ideal, and we are no better off than before. What we need is a better generating set/basis. \\

In the language of ideals, what was the problem with the example shown earlier? When we use the order $(f_1, f_2)$, we obtain $-x + y$ as a remainder. The problem is that we couldn't go any further by using the division algorithm. Say we somehow knew that
\[
-x + y = yf_1 - xf_2.
\]
Then we'd have no trouble deducing that our $f$ that we began with was indeed divisible by $\{f_1, f_2\}$. Our problem was that $LT(-x+y)$ was not divisible by either $LT(f_1)$ or $LT(f_2)$. 

\begin{mdframed}
\textcolor{Violet}{\textbf{Note:} $-x + y$ was formed with multiplying both $f_1$ and $f_2$ with monomials so as to form the LCM of the leading terms of $f_1$ and $f_2$, and then cancelling it out. In fact, the notion of an S-polynomial/S-pair comes from this very observation, and in turn leads us to Buchberger's algorithm.}
\end{mdframed}

\begin{ideabox}
\label{idea:gb}
This leads us to realize that we need a generating set $\{g_i\}_{i \in [s]}$ of our ideal $I = \ideal{f_1, f_2}$ that is such that for any $f \in I$, we have that there exists an $i^* \in [s]$ such that $LT(g_{i^*}) \mid LT(f)$.
\end{ideabox}

Idea \ref{idea:gb} is basically the definition of a \gb basis. A \gb basis is defined exactly as what is needed according to Idea \ref{idea:gb}. By ensuring that we have a $g_{i^*}$ whose leading term divides $LT(f)$ for all $f \in I$, we are certain to always be able to get a remainder of $0$ when $f \in I$. This is because anything that is leftover after subtracting a multiple of $g_i$ from f is in the same congruence class of $f$, modulo $I$. \\

It is fairly reasonable to assume that one would get upto to this point, but how does one even hope that such a generating set would exist? How can we even hope that it would be finite? Keep in mind that Buchberger's algorithm actually tells you \emph{how to find it.} The transition from wondering if such a generating set could even exist to actually being able to always find it is possible because of an equivalent definition of a \gb basis.

\begin{definition}
\label{def:gb-as-ideals}
Define $LT(I) = \{LT(f) \mid f \in I\}$. The set $\{g_i\}_{i \in [s]} \subseteq I$ is a \gb basis if $I = \ideal{\{g_i\}_{i \in [s]}}$ and
\[
\ideal{LT(I)} = \ideal{\{LT(g_i) \mid i \in [s]\}}.
\]
\end{definition}

That this definition is equivalent to what we expressed in Idea \ref{idea:gb} is an exercise left to the reader. I am more interested in discussing how one would have arrived at this language. Apriori, I feel it is extremely unnatural to define $\ideal{LT(I)}$ and $\ideal{\{LT(g_i) \mid i \in [s]\}}$. How then? \\

Remember, we want, for all $f \in I$, $LT(f) = m*LT(g_i)$ for some $i$ and $m$ a monomial. So naturally we would take a look at the set $LT(I)$. Let us form all possible $m*LT(g_i)$. It is just the set
\[
\{LT(f) * LT(g_i) \mid i \in [s], f \in k[x_1, \ldots, x_n]\}.
\]
We want the above set to contain $LT(I)$, i.e. we want $\{LT(g_i)LT(f) \mid i \in [s], f \in k[x_1, \ldots, x_n]\} \supseteq LT(I)$. Since $g_i \in I$, we have that $g_iLT(f) \in I$ for all $f \in k[x_1, \ldots, x_n]$. This in turn means that $LT(g_i)LT(f)$ will be in $LT(I)$ thus proving that $\{LT(g_i)LT(f) \mid i \in [s], f \in k[x_1, \ldots, x_n]\} \subseteq LT(I)$. Thus we can say that the necessity expressed in Idea \ref{idea:gb} is equivalent to needing
\[\{LT(g_i)LT(f) \mid i \in [s], f \in k[x_1, \ldots, x_n]\} = LT(I),\]
as sets. Since we need them to be the same as sets, we can just say that we need
\[\ideal{\{LT(g_i)LT(f) \mid i \in [s], f \in k[x_1, \ldots, x_n]\}} = \ideal{LT(I)}.\] Finally, $\ideal{\{LT(g_i)LT(f) \mid i \in [s], f \in k[x_1, \ldots, x_n]\}}$ may as well be written as $\ideal{\{LT(g_i) \mid i \in [s]\}}$, thus giving us the condition in Definition \ref{def:gb-as-ideals}. \\

N.B. Actually the above discussion only proves that $\ideal{LT(I)} = \ideal{\{LT(g_i) \mid i \in [s]\}}$ is a sufficient condition for what is expressed in Idea \ref{idea:gb}. It is also a necessary condition, but that needs a proof. Proving it is a necessary condition requires the observation that if we have a monomial $m \in I = \ideal{\{m_i\}}$, where $m_i$ are also monomials, then there exists an $i^*$ such that $m_{i^*} \mid m$. \\

The advantage of moving to the new language to define \gb bases is that $\ideal{LT(I)}$ is now a monomial ideal, and we have results like Dickson's lemma that show to us that \gb bases will always exist and also show us how to find them.
%\bibliographystyle{plainnat}
%\bibliography{main} 
\end{document}